%-----------------------------------------------------------------------------%
\chapter*{\kataPengantar}
%-----------------------------------------------------------------------------%
Puji syukur kehadirat Allah SWT yang telah melimpahkan rahmat, nikmat, dan hidayah-Nya sehingga \saya~dapat memiliki kekuatan dan pengetahuan untuk merampungkan tugas akhir yang berjudul "\judul" ini. Tanpa ilmu dan izin yang diberikan-Nya, \saya~tidak akan mampu menyelesaikan tugas akhir ini. Shalawat serta salam juga senantiasa \saya~haturkan kepada Nabi Muhammad SAW. Tugas akhir ini juga berhasil \saya~selesaikan atas bantuan dan dukungan dari berbagai pihak. Untuk itu, penulis mengucapkan terima kasih kepada:
\begin{enumerate}
	\item Ibu dan Ayah penulis, Nafisah dan Said Sungkar, yang senantiasa mendoakan \saya, memberikan berbagai dukungan, dan menjadi motivasi \saya~untuk menyelesaikan tugas akhir ini tepat waktu. Tiga adik \saya, Sarah, Yasmin, dan Nada Sungkar yang selalu menghibur \saya~dengan berbagai caranya masing-masing ketika \saya~berada di rumah.
	\item Bapak Bob Hardian, Ph.D yang telah menyumbangkan waktu dan pengetahuannya untuk membimbing \saya~dalam mengerjakan tugas akhir ini.
	\item Rekan-rekan satu kelompok topik \textit{social internet of things}, Abdullah Izzuddiin Alqassam, Jouvy Alif Pradewo, Mario Joel, Muhammad Redho Ayassa, dan Prakoso Adi Nugroho yang telah berjuang bersama dan saling membantu selama pengerjaan tugas akhir ini.
	\item Teman-teman Lab Nokia yang terus menyemangati \saya~dengan caranya masing-masing dan menjadikan Lab Nokia tempat mengerjakan tugas akhir yang menyenangkan.
	\item Teman-teman Pushla, Ginanjar Ibnu Solikhin, Katri Adiningtyas, Laila Mauhibah, dan Rafi Putra Arriyan yang memberi pengalaman menyenangkan lain selama \saya~mengerjakan tugas akhir.
	\item Fauziah Rahmawati, Kurniagusta Dwinto, dan Riza Herzego Nida Fathan yang bersedia membantu \saya~dan membagikan ilmunya ketika \saya~mengalami kesulitan.
	\item Teman-teman Kawung 2011 yang selalu kompak dan selama 4 tahun terakhir terus membentuk \saya~menjadi pribadi yang lebih baik.
	\item Pihak-pihak lain yang membantu \saya~dalam berbagai hal dan tidak dapat \saya~sebutkan satu-persatu.
\end{enumerate}

Penulis berharap agar tugas akhir ini bisa memberi manfaat bagi sebanyak-banyaknya pihak dan bagi kemajuan perkembangan teknologi di Indonesia. Penulis menyadari bahwa tugas akhir ini masih terdapat kekurangan dan masih jauh dari sempurna. Oleh karena itu, \saya~mengharapkan kritik dan saran agar \saya~bisa menjadi lebih baik lagi di masa depan.

\vspace*{0.1cm}
\begin{flushright}
Depok, 14 Juni 2015\\[0.1cm]
\vspace*{1cm}
\penulis

\end{flushright}