%
% Halaman Abstract
%
% @author  Andreas Febrian
% @version 1.00
%

\chapter*{ABSTRACT}

\vspace*{0.2cm}

\noindent \begin{tabular}{l l p{11.0cm}}
	Name&: & \penulis \\
	Program&: & \program \\
	Title&: & \judulInggris \\
\end{tabular} \\ 

\vspace*{0.5cm}

\noindent 
Currently, there are various internet of things platform existed. One approach to make an internet of things platform is by using social internet of things concept that based on social media. To connect this platform with existing device, a "middle man" is needed to send information about the device to the platform and send command from the platform to the device. This "middle man" should be easily operated. In this final project, a gateway is implemented on Raspberry Pi to connect ZigBee based lights with a social internet of things platform. This gateway can be operated using a GUI. To send data from and to the gateway, a messaging protocol called MQTT is used. The MQTT topic used is designed so that different user connected to the platform can be identified. After several tests, the gateway device made in this final project work as expected in controlling lights using GUI as well as through MQTT message sent from the platform.\\

\vspace*{0.2cm}

\noindent Keywords: \\ 
\noindent Social Internet of Things Platform, Gateway, ZigBee, MQTT, Raspberry Pi

\newpage