%
% Halaman Abstrak
%
% @author  Andreas Febrian
% @version 1.00
%

\chapter*{Abstrak}

\vspace*{0.2cm}

\noindent \begin{tabular}{l l p{10cm}}
	Nama&: & \penulis \\
	Program Studi&: & \program \\
	Judul&: & \judul \\
\end{tabular} \\ 

\vspace*{0.5cm}

\noindent 
Saat ini, berbagai macam \plat~\iot~telah banyak bermunculan. Salah satu pendekatan untuk membuat \plat~ini adalah dengan menggunakan konsep \textit{social \iot} berbasis media sosial. Untuk bisa menghubungkan \plat~ini dengan perangkat yang ada, diperlukan sebuah perantara untuk mengirimkan informasi mengenai perangkat yang ada ke \plat~dan untuk mengirimkan perintah dari \plat~ke perangkat. Selain itu, perangkat tersebut juga perlu bisa dioperasikan dengan mudah. Pada tugas akhir ini diimplementasikan sebuah perangkat \textit{gateway} menggunakan Raspberry Pi untuk menghubungkan perangkat lampu berbasis ZigBee dengan sebuah \plat~social \iot~dan dapat dioperasikan menggunakan GUI. Untuk mengirimkan data dari dan ke perangkat \textit{gateway}, digunakan sebuah protokol pengiriman pesan, yaitu MQTT. Topik MQTT yang digunakan didesain agar dapat mengenali identitas pengguna berbeda yang akan terhubung ke \plat. Setelah dilakukan beberapa pengujian, perangkat \textit{gateway} yang dibuat dapat berjalan sesuai ekspektasi, baik dalam melakukan kontrol melalui GUI yang disediakan maupun melalui pesan MQTT yang dikirimkan dari \plat.\\

\vspace*{0.2cm}

\noindent Kata Kunci: \\ 
\noindent \textit{Social Internet of Things Platform}, \textit{Gateway}, ZigBee, MQTT, Raspberry Pi\\

\newpage