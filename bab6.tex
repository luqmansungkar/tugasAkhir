%-----------------------------------------------------------------------------%
\chapter{\babEnam}
%-----------------------------------------------------------------------------%
Pada bab ini dijelaskan mengenai hasil pengujian, kesimpulan dari implementasi perangkat \textit{gateway} yang telah dilakukan, serta saran dari \saya~untuk pengembangan selanjutnya.

\section{Hasil Pengujian}
Dari hasil pengujian, dapat dilihat bahwa perangkat \textit{gateway} sudah dapat beroperasi dan digunakan sesuai dengan tujuan. Perangkat \textit{gateway} dapat berperan sebagai sebuah \textit{access point} secara otomatis setelah perangkat dinyalakan. Perangkat \textit{gateway} juga kemudian dapat dihubungkan dengan \textit{access point} lainnya melalui halaman aplikasi web yang disediakan. Operasi untuk mengontrol lampu melalui halaman aplikasi web yang disediakan juga dapat dilakukan dengan sukses. Setelah terhubung ke internet, pengguna juga dapat mendaftarkan lampu atau \textit{group} ke \plat~dengan sukses. Perintah yang dikirimkan oleh \textit{pub-sub system} ke topik yang sesuai juga berhasil diterima dan dikerjakan dengan sukses. Terakhir, perangkat \textit{gateway} berhasil mengirimkan informasi mengenai kondisi setiap lampu atau \textit{group} terdaftar ke \textit{pub-sub system} setiap jangka waktu tertentu.

\section{Kesimpulan}
Dalam tugas akhir ini, \saya~berhasil menghubungkan sebuah jaringan lampu berbasis ZigBee dengan sebuah \textit{platform} \iot~berbasis sosial media. Dalam tugas akhir ini, digunakan Raspberry Pi sebagai perangkat untuk menjalankan semua operasi terkait sehingga tidak diperlukan lagi komputer tambahan. Berdasarkan implementasi yang telah dilakukan, \saya~dapat mengambil beberapa kesimpulan, yaitu:
\begin{enumerate}
	\item sesuatu
\end{enumerate}